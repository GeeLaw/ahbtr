\subsubsection{Assumption.}
All of the primitives defined in this section are implied by the existence of
weakly selectively secure, single key, and sublinearly succinct
public-key functional encryption for general circuits
(so-called \emph{obfuscation-minimum PKFE}),
of which we refer the reader to~\cite{C:KNTY19} for the precise definition.

\begin{lemma}
Suppose there exists an \emph{obfuscation-minimum PKFE} with polynomial security, then there exist
\begingroup
\WideNarrow{
\def\MyCite#1{\leavevmode\unskip\rlap{\textup{#1}}\hspace*{13.6em}\ignorespaces}
}{
\def\MyCite#1{\leavevmode\unskip\rlap{\textup{#1}}\hspace*{5em}\ignorespaces}
}
\begin{itemize}
\item \MyCite{\cite{FOCS:Yao86,JC:LinPin09,CCS:BelHoaRog12}}
a secure circuit garbling scheme (Definitions~\ref{def:gc} and~\ref{def:gc-security}),
\item \MyCite{\cite{FOCS:GolGolMic84,AC:BonWat13,CCS:KPTZ13,PKC:BoyGolIva14}}
a secure PPRF (Definitions~\ref{def:pprf} and~\ref{def:pprf-security}),
\item \MyCite{\citeleft folklore\citeright}
a secure PKE scheme (Definitions~\ref{def:pke} and~\ref{def:pke-security}),
\item \MyCite{\cite{C:CDGGMP17,TCC:LiuZha17,TCC:AnaLom18,C:KNTY19}}
a secure laconic OT scheme (Definitions~\ref{def:lot} and~\ref{def:lot-security}), and
\item \MyCite{\cite{C:LinTes17,TCC:LiuZha17}}
an {\iO} for $\poly(\lambda)$-sized domain (Definitions~\ref{def:obfuscation} and~\ref{def:obfuscation-security}),
\end{itemize}
with polynomial security.
\endgroup
\end{lemma}
