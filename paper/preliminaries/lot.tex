\subsubsection{Laconic Oblivious Transfer.}
We rely on laconic oblivious transfer~\cite{C:CDGGMP17}.

\begin{definition}[LOT~\cite{C:CDGGMP17}]\label{def:lot}
A \emph{laconic oblivious transfer (OT) scheme} (with message space~$\bit^\lambda$) consists of 4 efficient algorithms:
\begin{itemize}
\item $\Gen(1^\lambda,{M\in\doubleN})$ takes the database length as input and outputs a hash key~$\hk$.
\item $\Hash(1^\lambda,\hk,{D\in\bit^M})$ takes as input a hash key and a database.
It is deterministic,
runs in time~$({M+1})\poly(\lambda,\log M)$, and
outputs a hash $h$ of length~$\poly(\lambda,\log M)$ and a processed database~$\widehat{D}$.
\item $\Send(1^\lambda,\hk,h,m\in[M],L_0\in\bit^\lambda,L_1\in\bit^\lambda)$
takes as input a hash key, a hash, an index, and two labels (messages).
It outputs a ciphertext~$\ct$.
\item $\Recv^{\widehat{D}}(1^\lambda,\hk,h,m\in[M],\ct)$
is given random access to a processed database, and
takes as input a hash key, a hash, an index, and a ciphertext.
It runs in time~$\poly(\lambda,\log M)$ and outputs a label (message).
\end{itemize}
The scheme must be \emph{correct}, i.e., for all
${\lambda\in\doubleN}$,
${M\in\doubleN}$,
${D\in\bit^M}$,
${m\in[M]}$,
${L_0,L_1\in\bit^\lambda}$,
\begin{align*}
\Pr\left[
\begin{aligned}
\hk&{}\draws\Gen(1^\lambda,M)\\
(h,\widehat{D})&{}\gets\Hash(1^\lambda,\hk,D)\\
\ct&{}\draws\Send(1^\lambda,\hk,h,m,L_0,L_1)
\end{aligned}
\::\:
\Recv^{\widehat{D}}(1^\lambda,\hk,h,m,\ct)=L_{D[m]}
\right]=1.
\end{align*}
\end{definition}
