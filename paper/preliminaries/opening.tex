We denote by~${\lambda\in\doubleN}$ the security parameter,
by $\poly({\cdot})$ a polynomial function, and
by $\negl(\lambda)$ a negligible function of~$\lambda$.
Efficient algorithms are probabilistic random-access machines $M^w(x)$ of running time~$\poly(|x|,|w|)$.
Efficient adversaries (in interactive experiments) are probabilistic Turing machines of (total) running time~$\poly(\lambda)$,
with or without $\poly(\lambda)$-long advices.
(All of the proofs in this work are uniform.)
The advantage of~$\scriptA$ in distinguishing $\Exp_0$ and~$\Exp_1$ is
${\Pr[\Exp_0^\scriptA(1^\lambda)=1]-\Pr[\Exp_1^\scriptA(1^\lambda)=1]}$.
We write $\approx,\approxS,\equiv$ for computational indistinguishability, statistical indistinguishability, and identity.

Under the standard assumption that a pseudorandom generator (with polynomial security) exists,
we can assume, whenever convenient, that a randomized algorithm uses a uniformly random $\lambda$-bit string as its randomness (without losing polynomial security considered in this work or degrading its efficiency).

For~${n,n'\in\doubleN}$, we write $[n..n']$ for the set~$\{n,\dots,n'\}$,
and $[n]$ for~$[1..n]$.
For a bit-string~$D$, we denote by $|D|$ its bit-length,
and given an index~${m\in[|D|]}$, we denote by~$D[m]$ the $m$\textsuperscript{th} bit of~$D$.
For two bit-strings~$D,D'$, their concatenation is~${D\concat D'}$.
Given a circuit ${C:\bit^{n+M_0}\to\bit^{n'}}$ and ${w\in\bit^n}$, we define $C[w]$ to be $C$ with $w$ hardwired as its first portion of input,
so~${C[w](x)=C(w\concat x)}$.
For an event~$X$, its indicator random variable is~$\1_X$.
For events~$X,Y$ in the same probability space, ``$X$ implies $Y$'' means ${X\subseteq Y}$.
