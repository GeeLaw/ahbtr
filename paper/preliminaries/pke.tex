\subsubsection{Public-Key Encryption.}
Our construction can be based on any public-key encryption scheme.

\begin{definition}[PKE]\label{def:pke}
A \emph{public-key encryption (PKE) scheme} (with message space~$\bit^\lambda$)
consists of 3 efficient algorithms:
\begin{itemize}
\item $\Gen(1^\lambda)$ outputs a pair $(\pk,\sk)$ of public and secret keys.
\item $\Enc(1^\lambda,\pk,m\in\bit^\lambda)$ takes as input the public key and a message.
It outputs a ciphertext~$\ct$.
\item $\Dec(1^\lambda,\sk,\ct)$ takes as input the secret key and a ciphertext.
It outputs a message.
\end{itemize}
The scheme must be \emph{correct}, i.e., for all
${\lambda\in\doubleN}$,
${m\in\bit^\lambda}$,
\begin{align*}
\Pr\left[
\begin{aligned}
(\pk,\sk)&{}\draws\Gen(1^\lambda)\\
\ct&{}\draws\Enc(1^\lambda,\pk,m)
\end{aligned}
\::\:
\Dec(1^\lambda,\sk,\ct)=m
\right]=1.
\end{align*}
\end{definition}

\begin{definition}[PKE security]\label{def:pke-security}
A PKE scheme $(\Gen,\Enc,\Dec)$ per Definition~\ref{def:pke} is \emph{semantically secure for random messages} (or \emph{secure} for the purpose of this work) if
\begin{align*}
\{(m_0,m_1,\pk,\ct_0)\}\approx\{(m_0,m_1,\pk,\ct_1)\},
\end{align*}
where $(\pk,\sk)\draws\Gen(1^\lambda)$ and
${m_b\draws\bit^\lambda}$, ${\ct_b\draws\Enc(1^\lambda,\pk,m_b)}$ for~${b\in\bit}$.
\end{definition}
