\begin{definition}[index-hiding]\label{def:ahplbe-index-hiding}
An AH-PLBE scheme $(\Gen,\Enc,\Dec)$ (Definition~\ref{def:ahplbe})
is \emph{index-hiding} if ${\ExpIH{0}\approx\ExpIH{1}}$,
where $\ExpIH{b}(1^\lambda)$ with adversary~$\scriptA$ proceeds as follows:
\begin{security}
\phase{Challenge}
Run
${(\pk,\sk)\draws\Gen(1^\lambda)}$,
launch $\scriptA(1^\lambda,\pk)$, and
receive from it
some~${N\in\doubleN}$,
a cut-off index~${i_\bot\in[N]}$, and
a list~$\{\pk_i\}_{i\in[N]\setminus\{i_\bot\}}$ of public keys.
Let ${\pk_{i_\bot}\gets\pk}$,
run
\begin{align*}
m\draws\bit^\lambda,\qquad
\ct\draws\Enc(1^\lambda,\{\pk_i\}_{i\in[N]},i_\bot-b,m),
\end{align*}
and send $(m,\ct)$ to $\scriptA$.
\phase{Guess}
$\scriptA$ outputs a bit~$b'$.
The output of the experiment is~$b'$ if
the keys in $\{\pk_i\}_{i\in[N]}$ are \emph{distinct}.
Otherwise, the output is set to~$0$.
\end{security}
\end{definition}
