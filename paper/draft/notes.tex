\section{Notes}

\begingroup
\color{red}
This is draft content.
Exclude \texttt{draft/notes} from \texttt{main.tex}.
\endgroup

\subsubsection{Literature.}
To be cited:
\begin{itemize}
\item (None at this time.)
\end{itemize}
To be investigated:
\begin{itemize}
\item \cite{CCS:WQZD10} introduces the notion of \ad hoc broadcast encryption.\\
See OneNote notes for its downsides and proposed improvements.
\item Adaptive security: It should be possible to choose whether or not to see $\mathsf{sk}$ depending on $\mathsf{pk}$ (in reality, the adversary could see everyone's public key and decide whose secret key to steal). This requires a ``Creation-Revelation'' type formulation of the game.
Has \cite{CCS:WQZD10} considered such case?
\item Can we do \ad hoc broadcast \textbf{and trace}?
\item In \ad hoc schemes (broadcast encryption or broadcast and trace), how to handle key pairs generated by the adversary itself?
Has \cite{CCS:WQZD10} considered such case?
For tracing, the algorithm should find out an identity in ${S_{\textup{challenge}}}\setminus S_{\textup{not queried}}$, where $S_{\textup{challenge}}$ is the identities to which the challenge ciphertext is encrypted and $S_{\textup{not queried}}$ is the identities whose secret keys are created but not revealed.
\item Investigate works in \texttt{crypto.bib} with ``\ad hoc'' in title.\\
Use \texttt{sls 'ad[ -]?hoc' crypto.bib} to obtain such works.
\end{itemize}
