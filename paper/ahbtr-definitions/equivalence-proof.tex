\begin{proof}[\hyperanchor{pf:ahbtr-security-equivalence}{Theorem~\ref{thm:ahbtr-security-equivalence}}]
The reductionist proof of necessity is straight-forward by setting ${Q=0}$ (resp.~${Q=1}$) for completeness (resp.~soundness).

To show sufficiency, suppose the AH-BTR scheme $(\Gen,\Enc,\Dec,\Trace)$ is both complete and sound and let $\scriptB$ be an efficient adversary against its traceability.
We consider two efficient adversaries.
$\scriptC_1$ is against the completeness of the scheme.
It works by internally simulating the traceability game for~$\scriptB$ and outputting (in the completeness experiment) whatever $\scriptB$ outputs.
Denoting probabilities and events for adversary~$\scriptX$ in its security experiment with subscript~$\scriptX$,
\begin{align*}
\GoodDist_{\scriptC_1}
\Longleftrightarrow
\GoodDist_\scriptB
\qquad\textup{and}\qquad
\NotFound_{\scriptC_1}
\Longleftrightarrow
\NotFound_\scriptB.
\end{align*}
Therefore,
\begin{align*}
\Pr_\scriptB[\GoodDist_\scriptB\wedge\NotFound_\scriptB]
=
\Pr_{\scriptC_1}
[\GoodDist_{\scriptC_1}\wedge\NotFound_{\scriptC_1}].
\end{align*}
$\scriptC_2$ is against the soundness of the scheme.
Let ${B=\poly(\lambda)>1}$ be an upper bound of~$Q$ that $\scriptB$ might ever output.
$\scriptC_2$ does the following:
\begin{itemize}
\item $\scriptC_2(\pk)$ launches~$\scriptB$, receives $1^Q$ from it,
samples and sets
\begin{align*}
s^\ast&{}\draws[B],&
\pk_{s^\ast}&{}\gets\pk,\\
S&{}\gets[Q],&
(\pk_q,\sk_q)&{}\draws\Gen()\quad\textup{for }q\in{[Q]\setminus{\{s^\ast\}}},
\end{align*}
and sends $\{\pk_q\}_{q\in[Q]}$ to~$\scriptB$.
\item $\scriptC_2$ answers queries from $\scriptB$ and updates $S$ as stipulated by the query phase of the traceability experiment, except that it aborts if $\scriptB$ queries for~$\sk_{s^\ast}$.
\item After the query phase,
$\scriptC_2$ checks whether ${\pk_s=\pk}$ for some ${s\in S}$ with ${s<s^\ast}$,
and aborts if so.
% The check above ensures the probability relation below is a strict equality,
% i.e.,
%     Pr_B  = B * Pr_(C_2)
% instead of
%     Pr_B <= B * Pr_(C_2).
% The idea is to ensure only one guess is "correct".
% Here, C_2 is guessing the smallest s* in S (after all the queries)
% such that pk*_(i*) = pk_(s*).
% Without this check, s* is not constrained to be the smallest,
% and there could be multiple "correct" s*'s.
% Although we can still show what we want, I personally find it
% aesthetically more pleasing to have precise equality relations.
Otherwise, $\scriptB$ outputs
\begin{align*}
\scriptD,\quad
\{\pk_j^\ast\}_{j\in[N]},\quad
1^{1/\epsilon^\ast},
\end{align*}
and $\scriptC_2$ looks for an%
\footnote{Such $i_\bot^\ast$, if it exists, is unique if $\GoodPKs_\scriptB$
(the interesting case).}
${i_\bot^\ast\in[N]}$ such that ${\pk_{i_\bot^\ast}^\ast=\pk}$.
If no such $i_\bot^\ast$ exists, $\scriptC_2$~aborts.
Otherwise, it outputs
\begin{align*}
\scriptD,\quad
N,\quad
i_\bot^\ast,\quad
\{\pk_j^\ast\}_{j\in{[N]\setminus{\{i_\bot^\ast\}}}},\quad
1^{1/\epsilon^\ast}.
\end{align*}
\end{itemize}
% This is a coupling argument.
Routine calculation yields
\begin{align*}
\Pr_\scriptB[\FalsePos_\scriptB]
=
B\Pr_{\scriptC_2}[\FalsePos_{\scriptC_2}].
\end{align*}
By the union bound,
\begin{align*}
&
\Pr_\scriptB[\FalsePos_\scriptB\vee(\GoodDist_\scriptB\wedge\NotFound_\scriptB)]
\\{}\leq{}&
\Pr_\scriptB[\FalsePos_\scriptB]
+
\Pr_\scriptB[\GoodDist_\scriptB\wedge\NotFound_\scriptB]
\\{}={}&
B\Pr_{\scriptC_2}[\FalsePos_{\scriptC_2}]
+
\Pr_{\scriptC_1}[\GoodDist_{\scriptC_1}\wedge\NotFound_{\scriptC_1}]
\WideNarrow{}{\hspace*{-1em}}
\\{}={}&
\poly(\lambda)\negl(\lambda)+\negl(\lambda)
=
\negl(\lambda).
\qedhere
\end{align*}
\end{proof}
