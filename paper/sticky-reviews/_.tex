This work was submitted to
Eurocrypt 2023 and TCC 2023.
This section includes select review paragraphs
that either have been addressed (by revision)
or should be clarified (by reply).
They are grouped by \textbf{topic},
marked with \underlined{venue/review}
(e.g., \underlined{EC/A} for review A from Eurocrypt 2023),
\texttt{quoted} then discussed.
Not in every case did we revise the text.

\subsubsection{AH-PLBE/BTR Security Notion Relations.}
From \underlined{EC/A}:

\texttt{
On page 15, when defining the security notions for AH-PLBE, the \\
authors mentioned "The two security definitions have a one-to-one \\
correspondence to the simplified security notions of AH-BTR in Sect. \\
3.1". Intuitively, is it correct if I understand this detail as "an \\
AH-PLBE is a particular version of AH-BTR, with no tracing algorithm"? \\
If not, how can we interpret this "one-to-one correspondence"?
}

We have revised the text, currently around the middle of page~15, to clarify
which security property of AH-PLBE translates to (is used to prove)
which of AH-BTR in the classic scheme
(Construction~\ref{con:ahbtr-from-ahplbe}).
The best way to think about this is that
``AH-PLBE vs.~AH-BTR'' is similar to ``KEM vs.~PKE''.
In both cases, the former formally is not a particular version of the latter
--- aside from tracing,
AH-PLBE encryption takes a cut-off index as input but AH-BTR does not;
KEM encapsulation outputs a random message, i.e., the encapsulated key,
but PKE encryption takes the actual message as input ---
yet in both cases, the former is a ``dumbed down'' version of the latter
and the security properties of the former can be used to prove
those of the latter in classic constructions.

\subsubsection{Syntax of Restricted BE.}
From \underlined{EC/A}:

\texttt{
In Def. 18 of Section 7 for restricted BE, the decryption algorithm \\
receives only the security parameter in unary and is having random \\
accesses to the public key as well as the secret key. At the same \\
time, the correctness is required to be perfect. Out of curiosity, \\
is it plausible to require perfect correctness when giving only \\
random accesses to the secret key? Anyway, the lower bound proven \\
in Section 7 still holds even when relaxing the correctness.
}

For the purpose of this lower bound,
it is fine to assume that each secret key of restricted BE
contains a verbatim copy of the master public key.
We did not do this because our definition format
(explicitly giving components to algorithms
instead of letting any component ``absorb'' any other)
is customary when efficiency is a concern,
even if the particular efficiency property is unaffected
by the change of definition.

\subsubsection{Lemma from Auxiliary-Input Random Oracle Model.}
From \underlined{EC/A}:

\texttt{
In the statement of Lemma 10, is it a typo in the constraint \\
of the presampling function G that \\
\hspace*{4em}"\#\{G(z,j) {\string\neq} {\string\bot}\} <= P for all z {\string\in} Z"? \\
Shouldn't it be \\
\hspace*{4em}\ \#\{G(z,R)[j] {\string\neq} {\string\bot}\} <= P for all z {\string\in} Z \\
\hspace*{18.5em}w.r.t a fixed R? \\
Moreover, it is not clear to me why the oracle algorithm B can be \\
inefficient. Could the authors please elaborate more on this point?%
}

We have revised the statement of Lemma~\ref{lem:ai-rom},
currently around the bottom of page~27.
We have revised the text, in the footnotes of page~28, to clarify
that $\scriptB$ does not have to be efficient for the lemma to hold,
although in our application, we invoke the lemma for an efficient~$\scriptB$.

\subsubsection{Encryption Randomness in Proof.}
From \underlined{EC/A}:

\texttt{
In the proof of Theorem 9, why does z contain the encryption's \\
randomness? It seems that this enc's randomness is never used by B\string^Y \\
(to recover ct it parses f).
}

It is a formalism issue.
Lemma~\ref{lem:ai-rom} is for ``function~$F$'',
so $F$ cannot use randomness.
The encryption randomness is included for use by~$F$.

\subsubsection{Restricted BE Security Definition.}
From \underlined{EC/A}:

\texttt{
In the security notion of restricted BE, at the beginning of \\
page 22, i* and mu\_0 are still contained in the {\string\cdots} in the \\
distribution on the right hand side of the {\string\approx} property, aren't \\
they?
}

We have revised the text, currently around the top of page~27,
to make the definition clear.
(Answer to original question: Yes, they are still included.)
