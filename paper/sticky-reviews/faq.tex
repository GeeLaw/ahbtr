This section is inspired by ``sticky reviews''
(\url{https://www.iacr.org/docs/author.pdf}).
Here, we respond to select questions/concerns raised by reviewers
of Euro\-crypt 2023 and TCC 2023.
Most other comments have been incorporated into the paper.

\subsubsection{Complex/Simple Definitions of Traceability.}
A reviewer wondered why we give
Definition~\ref{def:tracing-security} (full definition)
then Definitions~\ref{def:completeness} and~\ref{def:soundness}
(two equivalent smaller ones),
instead of just the latter two,
which would be equally convincing to the reviewer.

Since this work is studying a new notion and
security definitions are generally nuanced,
we derive Definition~\ref{def:tracing-security}
from \emph{first principles} ---
the full definition maps to our intuitive notion of security
in a straight-forward way.
After that, we study its simplification,
which might or might not be intuitive for security,
to different readers.

Another reviewer commented that
a previous version of Definition~\ref{def:tracing-security},
where the total number of ``new user initialization'' queries
was declared at the onset of the experiment,
was not good for intuition.
We revised the definition into its current form
in response to this other review.

\subsubsection{Strong Assumptions.}
Many reviewers raised the concern
that using obfuscation for AH-BTR appeared to be an overkill.

We would be thrilled to find a construction based on lighter hammers.
However, all other attempts till now have failed
to simultaneously achieve the desired properties ---
no global set-up,
robustly correct,
no bound on recipient list length,
more size-efficient than the na\"ive scheme.
We regard the construction using obfuscation as a good first step ---
doing so confirms that the notion is plausibly achievable and
the related notion of broadcast encryption has recently witnessed a great path
(\cite{C:BonWatZha14} first size-optimal scheme using obfuscation, then
\cite{EC:AgrYam20,TCC:AgrWicYam20} size-optimal from LWE and pairing, then
\cite{EC:Wee22} size-optimal from lattices).
Moreover, this work is not just about construction,
but also the formalization and
the complete characterization of Pareto frontier of efficiency.
