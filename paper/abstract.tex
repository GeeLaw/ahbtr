\begin{abstract}
Traitor tracing schemes (Chor--Fiat--Naor Crypto '94) help content distributors fight against piracy and
are defined with the content distributor as a trusted authority having access to the secret keys of all users.
While the traditional model caters well to its original motivation,
its centralized nature makes it unsuitable for many scenarios.
A notion of \ad hoc traitor tracing (naturally with the capability of broadcast and revocation) is proposed and studied in this work.
Such a scheme allows users in the system to generate their own public/secret key pairs, without trusting any other entity.
To encrypt, a list of public keys is used to identify the set of recipients, and decryption is possible with a secret key corresponding to any of the public keys in the list.
In addition, there is a tracing algorithm that given a list of recipients' public keys and a pirate decoder capable of decrypting ciphertexts encrypted to them, identifies at least one recipient whose secret key must have been used to construct the said decoder.

\abstractparindent
Two constructions are presented.
The first is based on obfuscation and has constant-size ciphertext, yet its decryption time is linear in the number of recipients.
The second is a generic transformation that reduces decryption time at the cost of increased ciphertext size.
A lower bound on the trade-off between ciphertext size and decryption time is shown, indicating that the two constructions achieve all possible optimal trade-offs.
The lower bound also applies to general attribute-based encryption.

% \keywords is optional (omit if not used).
\keywords{traitor tracing \and obfuscation \and attribute-based encryption.}

% \IACRCopyright is optional (omit if not used).
\IACRCopyright{%
This research is open-source.
See \url{https://github.com/GeeLaw/ahbtr}.
% {\textcopyright} {\color{red}Some Organization} 2021.
% This is the full version of a paper with the same title
% in the proceedings of {\color{red}Some Conference}
% published by {\color{red}Some Publisher}.
}
\end{abstract}
