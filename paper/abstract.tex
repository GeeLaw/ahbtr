\begin{abstract}
Traitor tracing schemes [Chor--Fiat--Naor, Crypto~'94] help content distributors fight against piracy and
are defined with the content distributor as a trusted authority having access to the secret keys of all users.
While the traditional model caters well to its original motivation,
its centralized nature makes it unsuitable for many scenarios.
For usage among mutually untrusted parties,
a notion of \ad hoc traitor tracing (naturally with the capability of broadcast and revocation) is proposed and studied in this work.
Such a scheme allows users in the system to generate their own public/secret key pairs, without trusting any other entity.
To encrypt, a list of public keys is used to identify the set of recipients, and decryption is possible with a secret key for any of the public keys in the list.
In addition, there is a tracing algorithm that given a list of recipients' public keys and a pirate decoder capable of decrypting ciphertexts encrypted to them, identifies at least one recipient whose secret key must have been used to construct the said decoder.

\abstractparindent
We present two constructions.
The first scheme relies on functional encryption for circuits (conceptually, obfuscation) and has constant-size ciphertext, yet its decryption time is linear in the number of recipients.
The second is a generic transformation that reduces decryption time,
yet its ciphertext size increases.
\luoji{To-do before submission:
Move contents, position floats, edit organization.
}
We prove a matching lower bound on the trade-off between ciphertext size and decryption time, showing that the two constructions achieve all possible optimal trade-offs, i.e., they fully demonstrate the Pareto front of efficiency.
The lower bound also applies to broadcast encryption (hence all mildly expressive attribute-based encryption schemes) and is of independent interest.

% \keywords is optional (omit if not used).
% \keywords{%
% \ad hoc \and
% decentralized \and
% distributed \and
% flexible \and
% traitor tracing \and
% broadcast encryption \and
% attribute-based encryption \and
% functional encryption \and
% obfuscation.}
% In this paper, the keywords should be removed for LNCS
% because the typesetting result is unsatisfactory
% and there is a limit of number of pages.

\keywordspdf{%
ad hoc;
decentralized;
distributed;
flexible;
traitor tracing;
broadcast encryption;
attribute-based encryption;
functional encryption;
obfuscation}

% \LaomianIacrCopyright is optional (omit if not used).
\LaomianIacrCopyright{%
% The ``open-thoughts'' statement should be typeset in LNCS using \thanks.
This research is ``open-thoughts''.
See \url{https://github.com/GeeLaw/ahbtr}.
% {\textcopyright} {\color{red}Some Organization} 2021.
% This is the full version of a paper with the same title
% in the proceedings of {\color{red}Some Conference}
% published by {\color{red}Some Publisher}.
}
\end{abstract}
