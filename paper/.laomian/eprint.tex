\documentclass[letterpaper,11pt,oneside,onecolumn,final]{article}

\def\widenarrow#1#2{#1}
\long\def\WideNarrow#1#2{#1}

% Use LNCS leveling:
\setcounter{secnumdepth}{2}
\def\subsubsection{\@startsection{subsubsection}{3}{\z@}%
{-12\p@ \@plus -4\p@ \@minus -4\p@}%
{-0.5em \@plus -0.22em \@minus -0.1em}%
{\normalfont\normalsize\bfseries\boldmath}}
\def\paragraph{\@startsection{paragraph}{4}{\z@}%
{-12\p@ \@plus -4\p@ \@minus -4\p@}%
{-0.5em \@plus -0.22em \@minus -0.1em}%
{\normalfont\normalsize\itshape}}

% Use LNCS captioning:
\long\def\@makecaption#1#2{%
  \small%
  \vskip\abovecaptionskip%
  \sbox\@tempboxa{{\bfseries #1.} #2}%
  \ifdim \wd\@tempboxa >\hsize%
    {\bfseries #1.} #2\par%
  \else%
    \global \@minipagefalse%
    \hb@xt@\hsize{\hfil\box\@tempboxa\hfil}%
  \fi%
  \vskip\belowcaptionskip%
}

% Supplement missing commands for metadata.
\let\titlerunning\Laomian@EatOneArgument
\let\orcidID\Laomian@EatOneArgument
\let\authorrunning\Laomian@EatOneArgument

\let\Laomian@Institute\relax
\def\inst#1{\unskip\textsuperscript{#1}}
\def\institute#1{\global\def\Laomian@Institute{#1}}
\def\email#1{\texttt{#1}}

\def\keywords#1{%
  \begingroup%
  \def\and{\unskip, }%
  \addvspace{2ex}\noindent\textbf{Keywords.} #1\@addpunct{.}\par%
  \endgroup%
}

% Create indentation for subsequent paragraphs of abstract (not needed).
\let\abstractparindent\relax

% Date the document.
\AtBeginDocument{\date{\ifcase\month%
  \or January\or February\or March\or April\or May \or June%
  \or July\or August\or September\or October\or November\or December%
  \fi\ \number\year%
}}

% \maketitle
\def\maketitle{%
  \begingroup%
  \null%
  \vskip 2em\relax%
  \begin{center}%
    {\LARGE\@title\par}%
    \vskip 1.5em\relax%
    {%
      \large\lineskip .5em\relax%
      \begin{tabular}[t]{c}%
        \@author%
      \end{tabular}%
      \par%
    }%
    \vskip 1em\relax%
    \begingroup%
    % Count the number of institutes.
    \setcounter{footnote}{1}%
    \newbox\Laomian@InstituteBox%
    \def\and{%
      \par\vskip 0.4em\relax%
      \noindent\stepcounter{footnote}%
      \textsuperscript{\arabic{footnote}}\,\ignorespaces%
    }%
    \setbox\Laomian@InstituteBox\vbox{\Laomian@Institute}%
    \par\noindent%
    \ifnum\c@footnote=1%
      % If there is only one institute, no need for the numbers.
      \setcounter{footnote}{0}%
    \else%
      % If there are 2 or more institutes,
      % they will be formatted like footnotes, and
      % "real" footnote numbering should continue from there.
      % This behavior comes from LNCS, but
      % it is suppressed if a separate title page is used.
      % See the definitions of \TypesetTableOfContents.
      \setcounter{footnote}{1}%
      \textsuperscript{1}\,\ignorespaces%
    \fi%
    \Laomian@Institute%
    \endgroup%
    \par\vskip 1em\relax%
    {\large \@date}%
  \end{center}%
  \par\vskip 1.2em\relax%
  \endgroup%
}

% Page layout.
\usepackage[letterpaper,textwidth=6in,textheight=9in,centering]{geometry}

% Show frames and create space for notes in draft mode.
\ifLaomianDraft@
\geometry{showframe}
\setlength\oddsidemargin{-0.8in}
\setlength\marginparwidth{2in}
\fi

% Page numbers are always enabled for ePrint.
\usepackage{lastpage}
\def\Laomian@PageStyle{%
  \pagenumbering{arabic}%
  \pagestyle{plain}%
  \def\@oddfoot{\hfil\thepage~/~\pageref*{LastPage}\hfil}
  \let\@evenfoot\@oddfoot
}
\def\Laomian@TitlePageStyle{%
  \pagenumbering{roman}%
  \pagestyle{plain}%
  \def\@oddfoot{\hfil\thepage\hfil}
  \let\@evenfoot\@oddfoot
}

\usepackage[utf8]{inputenc}
\usepackage[english]{babel}
\usepackage{xcolor}
\usepackage{float,graphicx}
\usepackage{tabularx,booktabs,multirow}
\usepackage{relsize}

\usepackage{afterpage}
\def\clearfloats{\afterpage{\clearpage}}

\usepackage[normalem]{ulem}
\usepackage[outline]{contour}

\def\ul#1{\uline{#1}}

\newif\ifLaomianInsideCallout@
\LaomianInsideCallout@false
\usepackage[backgroundcolor=white!0,linecolor=black!70,bordercolor=black!70,textsize=footnotesize]{todonotes}

% Prevent \subparagraph completely.
\let\subparagraph\Laomian@undefined

% Use pink page borders and notes.
\ifLaomianDraft@
\let\Laomian@Gm@pageframes@bak\Gm@pageframes
\def\Gm@pageframes{\begingroup\color{red!50}\Laomian@Gm@pageframes@bak\endgroup}
% Rotate among these colors for call-outs.
\definecolor{LaomianCalloutColorA}{rgb}{0.965,0.365,0.208}
\definecolor{LaomianCalloutColorB}{rgb}{0.506,0.733,0.020}
\definecolor{LaomianCalloutColorC}{rgb}{1.000,0.729,0.027}
\definecolor{LaomianCalloutColorD}{rgb}{0.016,0.647,0.941}
\def\Laomian@CalloutColorName{LaomianCalloutColorA}
\def\Laomian@CalloutColorNameA{LaomianCalloutColorB}
\def\Laomian@CalloutColorNameB{LaomianCalloutColorC}
\def\Laomian@CalloutColorNameC{LaomianCalloutColorD}
\def\Laomian@CalloutColorNameD{}
\def\Laomian@CalloutColorRotate{%
  \global\let\Laomian@CalloutColorNameD\Laomian@CalloutColorName%
  \global\let\Laomian@CalloutColorName\Laomian@CalloutColorNameA%
  \global\let\Laomian@CalloutColorNameA\Laomian@CalloutColorNameB%
  \global\let\Laomian@CalloutColorNameB\Laomian@CalloutColorNameC%
  \global\let\Laomian@CalloutColorNameC\Laomian@CalloutColorNameD%
}
\def\defcallout#1#2{\def#1##1{%
  \ifLaomianInsideCallout@%
    \\\textbf{#2:} ##1%
  \else%
    \LaomianInsideCallout@true
    \leavevmode%
    \todo[%
      linecolor=\Laomian@CalloutColorName,%
      bordercolor=\Laomian@CalloutColorName%
    ]{\textbf{#2:} ##1}%
    \LaomianInsideCallout@false%
    \Laomian@CalloutColorRotate%
    \ignorespaces%
  \fi%
}}
\else
\def\defcallout#1#2{\def#1##1{\ignorespaces}}
\fi
\AtBeginDocument{\let\defcallout\Laomian@undefined}

% Anonymity (\TypesetAcknowledgement and \maketitle).
% Also handles page style for \maketitle.
\ifLaomianAnonymous@
\let\TypesetAcknowledgement\relax
\AtBeginDocument{%
  % Back up the title commands.
  \let\Laomian@author@bak\author
  \let\Laomian@authorrunning@bak\authorrunning
  \let\Laomian@institute@bak\institute
  \let\Laomian@maketitle@bak\maketitle
  % Replace the authorship commands.
  \let\author\Laomian@EatOneArgument%
  \let\authorrunning\Laomian@EatOneArgument%
  \let\institute\Laomian@EatOneArgument%
  % Restore the authorship commands right before \maketitle.
  \def\maketitle{%
    \let\author\Laomian@author@bak%
    \let\authorrunning\Laomian@authorrunning@bak%
    \let\institute\Laomian@institute@bak%
    \let\maketitle\Laomian@maketitle@bak%
    \let\Laomian@author@bak\Laomian@undefined%
    \let\Laomian@authorrunning@bak\Laomian@undefined%
    \let\Laomian@institute@bak\Laomian@undefined%
    \let\Laomian@maketitle@bak\Laomian@undefined%
    \author{}%
    \authorrunning{}%
    \institute{}%
    \maketitle%
    \Laomian@TitlePageStyle%
    \vspace*{-7.5ex}%
  }%
}
\else
\def\TypesetAcknowledgement{\subsubsection{Acknowledgement.}
The author was supported by NSF grants CNS-1936825 (CAREER), CNS-2026774, a J.P.~Morgan AI Research Award, and a Simons Collaboration on the Theory of Algorithmic Fairness.
The views expressed in this work are those of the author and do not reflect the official policy or position of any of the supporters.
He started researching this topic after rereading a post~\cite{V2EX:759538} on V2EX.
}
\AtBeginDocument{%
  \let\Laomian@maketitle@bak\maketitle%
  \def\maketitle{%
    \let\maketitle\Laomian@maketitle@bak%
    \let\Laomian@maketitle@bak\Laomian@undefined%
    \maketitle%
    \Laomian@TitlePageStyle%
  }%
}
\fi

% Page limits.
\ifLaomianPageLimits@
\def\pagelimitsnolimits#1#2{#1}
\long\def\PageLimitsNoLimits#1#2{#1}
\else
\def\pagelimitsnolimits#1#2{#2}
\long\def\PageLimitsNoLimits#1#2{#2}
\fi

% IACR copyright notice.
\ifLaomianCopyright@
\def\IACRCopyright#1{%
  \begingroup%
  \def\@makefntext##1{\noindent ##1}%
  \footnotetext{#1}%
  \endgroup%
}
\else
\let\IACRCopyright\Laomian@EatOneArgument
\fi

% Table of contents (default).
\let\TypesetTableOfContents\relax
\let\Laomian@BibInTOC@Begin\relax
\let\Laomian@BibInTOC@End\relax
\let\Laomian@AppendixInTOC\relax

% Bibliography.
\AtBeginDocument{\def\doi#1{\url{https://doi.org/#1}}}
\ifLaomianNumericBib@
\def\bibalphanumeric#1#2{#2}
\long\def\BibAlphaNumeric#1#2{#2}
\usepackage[noadjust,nomove]{cite}
\def\TypesetBibliography#1{%
  \bibliographystyle{.laomian/splncs04}%
  \Laomian@BibInTOC@Begin%
  \bibliography{#1}%
  \Laomian@BibInTOC@End%
}
\else
\def\bibalphanumeric#1#2{#1}
\long\def\BibAlphaNumeric#1#2{#1}
\usepackage[noadjust,nomove,nosort,nocompress]{cite}
\def\TypesetBibliography#1{%
  \bibliographystyle{alpha}%
  \Laomian@BibInTOC@Begin%
  \bibliography{#1}%
  \Laomian@BibInTOC@End%
}
\fi

\mathchardef\citeprepenalty=0
\def\citeleft{[}
\def\citeright{]}
\def\citepunct{,\hspace{0pt}}
\def\citedash{--\penalty10000\relax}
\def\citemid{; }

% Appendix (supplementary materials).
\ifLaomianSupplementary@
\def\Laomian@AppendixName{Supplementary Materials}
\def\Laomian@AppendixTitle{\begin{center}\huge\bfseries\Laomian@AppendixName\end{center}}
\else
\def\Laomian@AppendixName{Appendix}
\let\Laomian@AppendixTitle\relax
\fi

\def\TypesetAppendix{%
  \clearpage%
  \Laomian@AppendixInTOC%
  \Laomian@AppendixTitle%
  \appendix%
}

% Security experiments.
\def\security{%
  \begin{itemize}%
  \upshape%
}
\def\endsecurity{\end{itemize}}
\def\phase#1{\item{\bfseries\upshape #1.}}%


% Underlines.
\def\ULdepth{0.45ex}
\contourlength{0.32ex}
\def\ul#1{\uline{\phantom{\contour*{white}{#1}}}\llap{\contour*{white}{#1}}}

% Reference names
\def\Figure{Figure}
\def\Equation{Equation}
\def\Section{Section}
\def\Figures{Figures}
\def\Equations{Equations}
\def\Sections{Sections}

\ifLaomianLinkColors@
\definecolor{LaomianCiteColor}{rgb}{0,0,0.6}
\definecolor{LaomianLinkColor}{rgb}{0.8,0,0}
\definecolor{LaomianURLColor}{rgb}{0,0.3,0.8}
\usepackage[colorlinks=true,citecolor=LaomianCiteColor,linkcolor=LaomianLinkColor,filecolor=LaomianURLColor,urlcolor=LaomianURLColor]{hyperref}
\else
\usepackage[colorlinks=true,allcolors=blue]{hyperref}
\fi

% Implement ORCID icon.
\ifLaomianORCIDIcon@
\usepackage{orcidlink}
\def\orcidID#1{\unskip\,\orcidlink{#1}}
\fi


\let\proof\Laomian@undefined
\let\endproof\Laomian@undefined
\usepackage{amsfonts,amsmath,amsthm,amssymb,mathtools}
\usepackage{bbm,dsfont,mathrsfs,stmaryrd}
\usepackage{dashbox}

\def\thmhead@plain#1#2#3{%
  \thmname{#1}\thmnumber{\@ifnotempty{#1}{ }\relax{#2}}%
  \thmnote{ {\the\thm@notefont(#3)}}%
}
\let\thmhead\thmhead@plain

% Undefine these theorem environments as well as the counters.
% They are redefined later.
\def\Laomian@undefine@theorem#1{%
  \expandafter\let\csname #1\endcsname\Laomian@undefined%
  \expandafter\let\csname end#1\endcsname\Laomian@undefined%
  \expandafter\let\csname c@#1\endcsname\Laomian@undefined%
}

\Laomian@undefine@theorem{assumption}
\Laomian@undefine@theorem{claim}
\Laomian@undefine@theorem{conjecture}
\Laomian@undefine@theorem{construction}
\Laomian@undefine@theorem{corollary}
\Laomian@undefine@theorem{definition}
\Laomian@undefine@theorem{lemma}
\Laomian@undefine@theorem{proof}
\Laomian@undefine@theorem{proposition}
\Laomian@undefine@theorem{question}
\Laomian@undefine@theorem{remark}
\Laomian@undefine@theorem{theorem}

% Undefine these theorem environments but keep the counters.
% They are not redefined later.
% Their counters are reset by LNCS and must not be undefined.
\def\Laomian@undefine@theorem#1{%
  \expandafter\let\csname #1\endcsname\Laomian@undefined%
  \expandafter\let\csname end#1\endcsname\Laomian@undefined%
}

\Laomian@undefine@theorem{case}
\Laomian@undefine@theorem{example}
\Laomian@undefine@theorem{exercise}
\Laomian@undefine@theorem{note}
\Laomian@undefine@theorem{problem}
\Laomian@undefine@theorem{property}
\Laomian@undefine@theorem{solution}

\let\Laomian@undefine@theorem\Laomian@undefined

% Modified from amsthm.sty:
\def\th@remark{%
  \thm@headfont{\itshape}%
  \normalfont % body font
  % Use normal spacing.
  % \thm@preskip\topsep \divide\thm@preskip\tw@
  % \thm@postskip\thm@preskip
}

\theoremstyle{remark}
\newtheorem{remark}{Remark}
\newtheorem*{remarks}{Remarks}

\newtheorem*{Laomian@proof}{Proof}
\def\proof{\@ifnextchar[\Laomian@proof@with@name\Laomian@proof@without@name}
\def\Laomian@proof@with@name[#1]{%
  \begingroup%
  \pushQED{\qed}%
  \begin{Laomian@proof}[#1]%
}
\def\Laomian@proof@without@name{%
  \begingroup%
  \pushQED{\qed}%
  \begin{Laomian@proof}%
}
\def\endproof{%
  \popQED%
  \end{Laomian@proof}%
  \endgroup%
}

\ifLaomianLNCS@
\theoremstyle{plain}
\else
\theoremstyle{definition}
\fi
\newtheorem{definition}{Definition}
\newtheorem{question}{Question}

\theoremstyle{plain}
\newtheorem{conjecture}{Conjecture}
\newtheorem{corollary}{Corollary}
\newtheorem{claim}{Claim}
\newtheorem{lemma}{Lemma}
\newtheorem{proposition}{Proposition}
\newtheorem{theorem}{Theorem}

\let\c@corollary\c@theorem
\let\c@claim\c@theorem
\let\c@lemma\c@theorem
\let\c@proposition\c@theorem

\newtheorem*{Laomian@restated}{\Laomian@restated@name}
\def\restated#1{%
  \begingroup%
  \def\Laomian@restated@name{#1}%
  \begin{Laomian@restated}%
}
\def\endrestated{%
  \end{Laomian@restated}%
  \endgroup%
}

\theoremstyle{definition}
\newtheorem{assumption}{Assumption}
\newtheorem{construction}{Construction}


% Boxes.
\def\dashboxed#1{\dbox{\ensuremath{\displaystyle#1}}}
\definecolor{LaomianLightGray}{RGB}{200,200,200}
\def\grayboxed#1{\fcolorbox{LaomianLightGray}{LaomianLightGray}{\ensuremath{\displaystyle#1}}}

% Greek letters.
\DeclareMathSymbol{\Alpha}{\mathalpha}{operators}{"41}
\DeclareMathSymbol{\Beta}{\mathalpha}{operators}{"42}
\DeclareMathSymbol{\Gamma}{\mathalpha}{operators}{0}
\DeclareMathSymbol{\Delta}{\mathalpha}{operators}{1}
\DeclareMathSymbol{\Epsilon}{\mathalpha}{operators}{"45}
\DeclareMathSymbol{\Zeta}{\mathalpha}{operators}{"5A}
\DeclareMathSymbol{\Eta}{\mathalpha}{operators}{"48}
\DeclareMathSymbol{\Theta}{\mathalpha}{operators}{2}
\DeclareMathSymbol{\Iota}{\mathalpha}{operators}{"49}
\DeclareMathSymbol{\Kappa}{\mathalpha}{operators}{"4B}
\DeclareMathSymbol{\Lambda}{\mathalpha}{operators}{3}
\DeclareMathSymbol{\Mu}{\mathalpha}{operators}{"4D}
\DeclareMathSymbol{\Nu}{\mathalpha}{operators}{"4E}
\DeclareMathSymbol{\Xi}{\mathalpha}{operators}{4}
\DeclareMathSymbol{\Omicron}{\mathalpha}{operators}{"4F}
\DeclareMathSymbol{\Pi}{\mathalpha}{operators}{5}
\DeclareMathSymbol{\Rho}{\mathalpha}{operators}{"50}
\DeclareMathSymbol{\Sigma}{\mathalpha}{operators}{6}
\DeclareMathSymbol{\Tau}{\mathalpha}{operators}{"54}
\DeclareMathSymbol{\Upsilon}{\mathalpha}{operators}{7}
\DeclareMathSymbol{\Phi}{\mathalpha}{operators}{8}
\DeclareMathSymbol{\Chi}{\mathalpha}{operators}{"58}
\DeclareMathSymbol{\Psi}{\mathalpha}{operators}{9}
\DeclareMathSymbol{\Omega}{\mathalpha}{operators}{10}
\DeclareMathSymbol{\omicron}{\mathalpha}{letters}{"6F}

% Long live the orthodox!
\let\Laomian@epsilon@bak\epsilon
\let\epsilon\varepsilon
\let\varepsilon\Laomian@epsilon@bak
\let\Laomian@phi@bak\phi
\let\phi\varphi
\let\varphi\Laomian@phi@bak
\let\emptyset\varnothing
\let\varemptyset\varnothing
\let\nothing\varnothing

% Order symbols.
\def\bigO{\operatorname{O}}
\def\smallo{\operatorname{o}}
\def\bigTheta{\operatorname{\Theta}}
\def\bigOmega{\operatorname{\Omega}}
% \smallomega should be non-italicized, but
% I don't know a satisfactory solution for typesetting
% upright lower-case Greek letters in Computer Modern.
\def\smallomega{\operatorname{\omega}}

\def\cbrt{\sqrt[3]}
\def\vec#1{\boldsymbol{\mathbf{#1}}}
\def\1{{\mathbbm{1}}}

\def\esssup{\operatornamewithlimits{ess\,sup}}
\def\essinf{\operatornamewithlimits{ess\,inf}}
\def\EX{\operatornamewithlimits{\mathbb{E}}}
\def\Var{\operatorname{Var}}
\def\supp{\operatorname{supp}}

\def\approxS{\approx_{\textup{s}}}
\def\iseq{\overset{?}{=}}
\def\defeq{\overset{\mathsmaller{\textup{def}}}{=\joinrel=}}
\def\draws{\overset{\smash{\:_\$}\vphantom{{}_0}}{\leftarrow}}
\def\poly{\operatorname{poly}}
\def\negl{\operatorname{negl}}
\def\concat{{\parallel}}
\def\transpose{{\mathsmaller{\mathsf{T}}}}
\def\bit{{\{0,1\}}}

% 1..26 | % { [char](64 + $_) } | % { "\def\double$_{{\mathbb{$_}}}" }
\def\doubleA{{\mathbb{A}}}
\def\doubleB{{\mathbb{B}}}
\def\doubleC{{\mathbb{C}}}
\def\doubleD{{\mathbb{D}}}
\def\doubleE{{\mathbb{E}}}
\def\doubleF{{\mathbb{F}}}
\def\doubleG{{\mathbb{G}}}
\def\doubleH{{\mathbb{H}}}
\def\doubleI{{\mathbb{I}}}
\def\doubleJ{{\mathbb{J}}}
\def\doubleK{{\mathbb{K}}}
\def\doubleL{{\mathbb{L}}}
\def\doubleM{{\mathbb{M}}}
\def\doubleN{{\mathbb{N}}}
\def\doubleO{{\mathbb{O}}}
\def\doubleP{{\mathbb{P}}}
\def\doubleQ{{\mathbb{Q}}}
\def\doubleR{{\mathbb{R}}}
\def\doubleS{{\mathbb{S}}}
\def\doubleT{{\mathbb{T}}}
\def\doubleU{{\mathbb{U}}}
\def\doubleV{{\mathbb{V}}}
\def\doubleW{{\mathbb{W}}}
\def\doubleX{{\mathbb{X}}}
\def\doubleY{{\mathbb{Y}}}
\def\doubleZ{{\mathbb{Z}}}

% 1..26 | % { [char](64 + $_) } | % { "\def\script$_{{\mathcal{$_}}}" }
\def\scriptA{{\mathcal{A}}}
\def\scriptB{{\mathcal{B}}}
\def\scriptC{{\mathcal{C}}}
\def\scriptD{{\mathcal{D}}}
\def\scriptE{{\mathcal{E}}}
\def\scriptF{{\mathcal{F}}}
\def\scriptG{{\mathcal{G}}}
\def\scriptH{{\mathcal{H}}}
\def\scriptI{{\mathcal{I}}}
\def\scriptJ{{\mathcal{J}}}
\def\scriptK{{\mathcal{K}}}
\def\scriptL{{\mathcal{L}}}
\def\scriptM{{\mathcal{M}}}
\def\scriptN{{\mathcal{N}}}
\def\scriptO{{\mathcal{O}}}
\def\scriptP{{\mathcal{P}}}
\def\scriptQ{{\mathcal{Q}}}
\def\scriptR{{\mathcal{R}}}
\def\scriptS{{\mathcal{S}}}
\def\scriptT{{\mathcal{T}}}
\def\scriptU{{\mathcal{U}}}
\def\scriptV{{\mathcal{V}}}
\def\scriptW{{\mathcal{W}}}
\def\scriptX{{\mathcal{X}}}
\def\scriptY{{\mathcal{Y}}}
\def\scriptZ{{\mathcal{Z}}}

% 1..26 | % { [char](64 + $_) } | % { "\def\cursive$_{{\mathscr{$_}}}" }
\def\cursiveA{{\mathscr{A}}}
\def\cursiveB{{\mathscr{B}}}
\def\cursiveC{{\mathscr{C}}}
\def\cursiveD{{\mathscr{D}}}
\def\cursiveE{{\mathscr{E}}}
\def\cursiveF{{\mathscr{F}}}
\def\cursiveG{{\mathscr{G}}}
\def\cursiveH{{\mathscr{H}}}
\def\cursiveI{{\mathscr{I}}}
\def\cursiveJ{{\mathscr{J}}}
\def\cursiveK{{\mathscr{K}}}
\def\cursiveL{{\mathscr{L}}}
\def\cursiveM{{\mathscr{M}}}
\def\cursiveN{{\mathscr{N}}}
\def\cursiveO{{\mathscr{O}}}
\def\cursiveP{{\mathscr{P}}}
\def\cursiveQ{{\mathscr{Q}}}
\def\cursiveR{{\mathscr{R}}}
\def\cursiveS{{\mathscr{S}}}
\def\cursiveT{{\mathscr{T}}}
\def\cursiveU{{\mathscr{U}}}
\def\cursiveV{{\mathscr{V}}}
\def\cursiveW{{\mathscr{W}}}
\def\cursiveX{{\mathscr{X}}}
\def\cursiveY{{\mathscr{Y}}}
\def\cursiveZ{{\mathscr{Z}}}

% 1..26 | % { [char](64 + $_); [char](96 + $_); } | % { "\def\fraktur$_{{\mathfrak{$_}}}" }
\def\frakturA{{\mathfrak{A}}}
\def\fraktura{{\mathfrak{a}}}
\def\frakturB{{\mathfrak{B}}}
\def\frakturb{{\mathfrak{b}}}
\def\frakturC{{\mathfrak{C}}}
\def\frakturc{{\mathfrak{c}}}
\def\frakturD{{\mathfrak{D}}}
\def\frakturd{{\mathfrak{d}}}
\def\frakturE{{\mathfrak{E}}}
\def\frakture{{\mathfrak{e}}}
\def\frakturF{{\mathfrak{F}}}
\def\frakturf{{\mathfrak{f}}}
\def\frakturG{{\mathfrak{G}}}
\def\frakturg{{\mathfrak{g}}}
\def\frakturH{{\mathfrak{H}}}
\def\frakturh{{\mathfrak{h}}}
\def\frakturI{{\mathfrak{I}}}
\def\frakturi{{\mathfrak{i}}}
\def\frakturJ{{\mathfrak{J}}}
\def\frakturj{{\mathfrak{j}}}
\def\frakturK{{\mathfrak{K}}}
\def\frakturk{{\mathfrak{k}}}
\def\frakturL{{\mathfrak{L}}}
\def\frakturl{{\mathfrak{l}}}
\def\frakturM{{\mathfrak{M}}}
\def\frakturm{{\mathfrak{m}}}
\def\frakturN{{\mathfrak{N}}}
\def\frakturn{{\mathfrak{n}}}
\def\frakturO{{\mathfrak{O}}}
\def\frakturo{{\mathfrak{o}}}
\def\frakturP{{\mathfrak{P}}}
\def\frakturp{{\mathfrak{p}}}
\def\frakturQ{{\mathfrak{Q}}}
\def\frakturq{{\mathfrak{q}}}
\def\frakturR{{\mathfrak{R}}}
\def\frakturr{{\mathfrak{r}}}
\def\frakturS{{\mathfrak{S}}}
\def\frakturs{{\mathfrak{s}}}
\def\frakturT{{\mathfrak{T}}}
\def\frakturt{{\mathfrak{t}}}
\def\frakturU{{\mathfrak{U}}}
\def\frakturu{{\mathfrak{u}}}
\def\frakturV{{\mathfrak{V}}}
\def\frakturv{{\mathfrak{v}}}
\def\frakturW{{\mathfrak{W}}}
\def\frakturw{{\mathfrak{w}}}
\def\frakturX{{\mathfrak{X}}}
\def\frakturx{{\mathfrak{x}}}
\def\frakturY{{\mathfrak{Y}}}
\def\fraktury{{\mathfrak{y}}}
\def\frakturZ{{\mathfrak{Z}}}
\def\frakturz{{\mathfrak{z}}}


% Table of contents.
\ifLaomianTOC@
\usepackage{tocloft}
\setcounter{tocdepth}{2}
\def\cfttoctitlefont{\vspace*{1ex}\hfill\LARGE\bfseries}
\def\cftaftertoctitle{\hfill\vspace*{1ex}}
\def\cftsecleader{\hfil}
\def\cftsecfont{\bfseries}
\def\cftsecpagefont{\bfseries}
\setlength{\cftbeforesecskip}{0.4em}
\setlength{\cftbeforesubsecskip}{0.1em}
% Use alternative numbering for the title page and the table of contents.
\def\TypesetTableOfContents{%
  % Disable link coloring in the table of contents.
  \let\Laomian@TOC@Hy@colorlink@bak\Hy@colorlink%
  \def\Hy@colorlink##1{\begingroup}%
  % Print the table of contents.
  \newpage\tableofcontents%
  % Restore link coloring.
  \let\Hy@colorlink\Laomian@TOC@Hy@colorlink@bak%
  \let\Laomian@TOC@Hy@colorlink@bak\Laomian@undefined%
  % Restore numbering.
  % The footnote counter is reset even if there are multiple institutes.
  \newpage%
  \setcounter{page}{1}%
  \setcounter{footnote}{0}%
  \Laomian@PageStyle%
}
\def\Laomian@AppendixInTOC{%
  \addtocontents{toc}{%
    \protect\vskip 1.6em\protect\relax%
    {\protect\Large Appendix}%
    \protect\par%
  }%
}
\def\Laomian@BibInTOC@Begin{%
  % Back up \section.
  \let\Laomian@BibInTOC@section@bak\section%
  \def\section{\@ifstar\Laomian@BibInTOC@section\Laomian@BibInTOC@section}%
  \def\Laomian@BibInTOC@section##1{%
    % Produce \section*{References} and add it to the table of contents.
    \Laomian@BibInTOC@section@bak*{References}%
    \addcontentsline{toc}{section}{References}%
  }%
}
\def\Laomian@BibInTOC@End{%
  % Restore \section.
  \let\section\Laomian@BibInTOC@section@bak%
  \let\Laomian@BibInTOC@section@bak\Laomian@undefined%
  \let\Laomian@BibInTOC@section\Laomian@undefined%
}
\else
% Table of contents is hidden.
\ifLaomianTitlePage@
% Use separate title page without page number.
\def\Laomian@TitlePageStyle{%
  \pagenumbering{roman}%
  \pagestyle{empty}%
}
\def\TypesetTableOfContents{%
  % Restore numbering.
  % The footnote counter is reset even if there are multiple institutes.
  \newpage%
  \setcounter{page}{1}%
  \setcounter{footnote}{0}%
  \Laomian@PageStyle%
}
\else
% Do not use separate title page.
\let\Laomian@TitlePageStyle\Laomian@PageStyle
\Laomian@PageStyle
\fi
\fi

% Fonts.
\ifLaomianFonts@
\usepackage[default,lining,semibold,regular,scaled=0.9]{sourcecodepro}
\usepackage[default,lining,proportional,semibold,regular]{sourcesanspro}
\usepackage[default,lining,proportional,bold,regular]{sourceserifpro}
\usepackage[smallerops,varvw,varg,ncf,varbb,cmbraces,bigdelims,nonewtxmathopt,subscriptcorrection]{newtxmath}
\usepackage[cal=cm]{mathalfa}
% Adjust inter-word spacing for Source Serif Pro.
\AtBeginDocument{\relax\spaceskip=%
1.28\fontdimen2\font
plus 1.16666667\fontdimen3\font
minus 1\fontdimen4\font\relax}
% Adjust underline thickness to match Source Serif Pro.
\def\ULthickness{0.111111ex}
% Re-apply math-typesetting orthodoxy.
\let\emptyset\varnothing
\let\varemptyset\varnothing
\let\nothing\varnothing
\def\vec#1{\boldsymbol{\mathbf{#1}}}
% Use \upomega (from the fancy fonts) for \smallomega.
\def\smallomega{\operatorname{\upomega}}
% Match the new blackboard bold font.
\def\1{{\mathds{1}}}
\fi
