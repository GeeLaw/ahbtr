\begin{proof}[\hyperanchor{pf:ahbtr-from-ahplbe-badend}{Claim~\ref{clm:ahbtr-from-ahplbe-badend}}]
Given an efficient adversary~$\scriptB$ against the tracing security of Construction~\ref{con:ahbtr-from-ahplbe},
let ${B=\poly(\lambda)}$ be an upper bound of~$1/\epsilon^\ast$
($B$ exists since $1^{1/\epsilon^\ast}$ is output by~$\scriptB$ in polynomial time).
Consider the following efficient adversary~$\scriptA$ against the message-hiding property of~$\AHPLBE$:
\begin{itemize}
\item $\scriptA$ launches $\scriptB$ and simulates $\ExpTrace$ for it.
\item When $\scriptB$ outputs $\scriptD,\{\pk^\ast_j\}_{j\in[N]},1^{1/\epsilon^\ast}$,\WideNarrow{}{\rule[1.1em]{0pt}{0pt}}
the adversary~$\scriptA$ checks whether the public keys in $\{\pk^\ast_j\}_{j\in[N]}$ are distinct, and aborts if not.
\item If not aborting,\WideNarrow{}{\rule[0.9em]{0pt}{0pt}}
$\scriptA$ runs $\scriptE_N$ and
notes down ${\alpha\in\bit}$ indicating whether $E_N$ happened,
i.e., ${\alpha=1}$ if and only if $\scriptD$ guessed correctly in the trial.
\item $\scriptA$ submits $\{\pk^\ast_j\}_{j\in[N]}$ to the message-hiding experiment, receives $(\mu_0,\mu_1,\ct)$ back, and
runs and outputs ${b'\draws\scriptD(\mu_0,\mu_1,\ct)\oplus\alpha}$.
% This ``sign correction'', producing ``the wrong guess'',
% is sadly correct (making the advantage non-negative).
% This is because the advantage is
% \begin{align*}
%     \Pr[\Exp_0^\scriptA = 1] - \Pr[\Exp_1^\scriptA = 1],
% \end{align*}
% i.e., an adversary with positive advantage
% should say $1$ more often in~$\Exp_0$.
\end{itemize}
Routine calculation shows that the advantage of $\scriptA$ is
${\E[\1_\DistinctPKs\cdot 4\epsilon_N^2]}$,
which must be negligible by the message-hiding property of~$\AHPLBE$.
By Markov's inequality,
\begin{align*}
\Pr[\DistinctPKs\wedge\BadEnd]
&{}=
\Pr[\1_\DistinctPKs\cdot 4\epsilon_N^2>(\epsilon^\ast)^2]
\displaybreak[2]\\&{}\leq
\Pr[\1_\DistinctPKs\cdot 4\epsilon_N^2>B^{-2}]
\displaybreak[2]\\&{}\leq
B^2\E[\1_\DistinctPKs\cdot 4\epsilon_N^2]
\WideNarrow{}{\displaybreak[2]\\&{}}
=
(\poly(\lambda))^2\negl(\lambda)
=
\negl(\lambda).
\qedhere
\end{align*}
\end{proof}
