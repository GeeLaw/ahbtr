\begin{definition}[AH-BTR]\label{def:ahbtr}
An \emph{\ad hoc broadcast, trace, and revoke (AH-BTR) scheme}
(with message space~$\bit^\lambda$ and public key length~$\ell(\lambda)$)
consists of 4 efficient algorithms:
\begin{itemize}
\item $\Gen(1^\lambda)$ outputs a pair $(\pk,\sk)$ of public and secret keys
with ${|\pk|=\ell(\lambda)}$.
\item $\Enc(1^\lambda,\{\pk_j\}_{j\in[N]},{\mu\in\bit^\lambda})$
takes as input
\luoji{See Definition~\ref{def:ahplbe} for \textit{\textbf{distinct}}.}
a list of \textit{\textbf{distinct}} public keys and a message.
It outputs a ciphertext~$\ct$.
\item $\Dec^{\{\pk_j\}_{j\in[N]},\ct}(1^\lambda,N,i\in[N],\sk_i)$
is given random access to a list of public keys and a ciphertext, and
takes as input
the length of the list,
an index, and
a secret key.
It outputs a message.
\item $\Trace^\scriptD(1^\lambda,\{\pk^\ast_j\}_{j\in[N]},1^{1/\epsilon^\ast})$
is given oracle access to a (stateless randomized) distinguisher~$\scriptD$ and takes as input
a list of \textit{\textbf{distinct}} public keys and an error bound.
It outputs an index~${i^\ast\in\{\bot\}\cup[N]}$.
\end{itemize}
The scheme must be \emph{robustly correct}, i.e., for all
${\lambda\in\doubleN}$,
${N\in\doubleN}$,
${i\in[N]}$,\WideNarrow{}{\linebreak[4]}
$\{\pk_j\}_{j\in[N]\setminus\{i\}}$%
\footnote{These public keys could be out of the support of~$\Gen$,
i.e., malformed.}
such that ${|\pk_j|=\ell(\lambda)}$ for all ${j\in[N]\setminus\{i\}}$, and
${\mu\in\bit^\lambda}$,
\begin{align*}
\Pr\left[
\begin{aligned}
(
\WideNarrow{}{{}&}
\pk_i,\sk_i)
\WideNarrow{&{}}{}
\draws\Gen(1^\lambda)\\
\WideNarrow{}{&}
\ct
\WideNarrow{&{}}{}
\draws\Enc(1^\lambda,\{\pk_j\}_{j\in[N]},\mu)
\end{aligned}
\::\:
\begin{aligned}
\Dec^{\{\pk_j\}_{j\in[N]},\ct}(1^\lambda,N,i,\sk_i)&{}=\mu
\\\textup{or }\exists\, j,j'\in[N]:\WideNarrow{\,\ }{}
(j\neq j')\wedge(\pk_j&{}=\pk_{j'})
\end{aligned}
\right]=1.
\end{align*}
\end{definition}
