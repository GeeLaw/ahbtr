\subsubsection{Simplifying Security Notions.}
Traditionally~\cite{EC:BonSahWat06}, traceability has been defined using one comprehensive \emph{interactive} experiment,%
% The definition of TCC:CVWWW18 seems to have a problem...
% In the first version of ePrint 2018/897, on p.19, the definition of
% tracing security requires
%     Pr[GoodDecoder && BadTrace] = negligible,
% which does not (from a definitional point of view) prevent Trace
% from accusing an innocent user if D has advantage 0.
% The black-box confirmation model (where there is separate
% confirmation/soundness notion) is not as related as I anticipated,
% and mentioning here should be good enough.
\footnote{While some previous works~\cite{C:BonFra99,STOC:GoyKopWat18,C:Zhandry20} separate traceability into multiple notions,
each notion still requires interaction in its security experiment,
due to the centralized nature of
the set-up process of traditional traitor tracing.}
which is complicated to work with.
Intuitively, the notion requires that
\emph{i)}~a traitor should be found from a decoder with sufficient advantage and
\emph{ii)}~no honest user should be identified as a traitor, regardless of the decoder advantage.

We thus define two security notions for AH-BTR capturing the requirements separately.
The former is called \emph{completeness} and the latter is called \emph{soundness}.
Their conjunction is equivalent to \emph{traceability}.
Since only one requirement is considered in each notion,
both of them can be vastly simplified and
the security experiments become \emph{non-interactive}.
They are much more convenient for reductionist proofs.
