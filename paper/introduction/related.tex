\subsubsection{Related Works.}
We discuss several related works.

\paragraph{\Ad Hoc, Decentralized, Distributed, or Flexible
Broadcast Encryption.}
\leavevmode\unskip\footnote{
While the details of the definition in each work differ,
their common theme is that each recipient generates their own key pair.}
Decentralized BE
with interactive management of recipient sets
was studied in~\cite{SCN:PhaPoiStr12,PAIRING:DelPaiPoi07}.
\Ad hoc (a.k.a.~distributed or flexible) BE
was studied in several prior/later works.
% It's an open problem to construct unbounded schemes
% from pairing with non-trivial storage efficiency.
Schemes based on
pairing~\cite{DHMR08,CCS:WQZD10,KolMalWee23} or
% This construction is bounded because its security definition
% constrains "n" (binary, maximum number of recipients, given to Setup)
% to be polynomial.
% It's unclear to me whether the restriction can be removed with some effort,
% e.g., by considering "function-binding hash" for sparse strings.
witness encryption~\cite{C:FreWatWu23}
require global set-up, and
% This construction is bounded because of its syntax/security definition.
the obfuscation-based one~\cite{C:BonZha14} do not.
% However, nowhere else in this paper did I mention the distinction between
% bounded and unbounded schemes, so I am not mentioning it here either.

\paragraph{Efficiency Parameters.}
Existing works studying
BE~\cite{C:FiaNao93,C:BonGenWat05,EC:GenWat09,C:BonWatZha14,EC:AgrYam20,TCC:AgrWicYam20,EPRINT:BraVai20,C:Zhandry20,EC:Wee22}
and
its extensions~\cite{PAIRING:DelPaiPoi07,AC:Delerablee07,EPRINT:SakFur07,C:BonZha14}
have been focused on improving the sizes of various components,
and the time complexity has been largely overlooked.
In a rare exception,
the work of~\cite{PKC:AttLib10}
reduces the number of pairing operations during decryption down to constant,
yet the overall decryption time is not among its concerns.
This work brings the total decryption time into the picture.

\paragraph{Lower Bounds.}
Previous works~\cite{EC:BluCre94,EC:LubSta98,AC:KYDB98,AFRICACRYPT:AusKre08,AC:KatYer09,C:GayKerWee15,ITC:DLY21} show a few efficiency lower bounds related to ABE and BE,
yet they only apply to information-theoretically secure primitives and even specific construction techniques.
Moreover, all of them prove space (ciphertext or secret key size, or their trade-off) lower bounds, whereas
this work is about space-time trade-offs.
Based on obfuscation~\cite{C:BonWatZha14} or both LWE and pairing~\cite{EC:AgrYam20}, broadcast encryption with ${|\ct|,|\sk|=\bigO(1)}$ can be achieved,
circumventing all previously known bounds.
In a concurrent work~\cite{EC:JaiLinLuo23},
lower bounds of (partially hiding) functional encryption are shown.
