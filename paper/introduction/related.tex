\subsubsection{Related Works.}
We discuss them by aspects.

\paragraph{\Ad Hoc, Decentralized, Distributed, or Flexible
Broadcast Encryption.}
\leavevmode\unskip\footnote{
While the details of the definition in each work differ,
their common theme is that each recipient generates their own key pair.}
Decentralized BE
with interactive management of recipient sets
was studied in~\cite{SCN:PhaPoiStr12,PAIRING:DelPaiPoi07}.
\Ad hoc (also known as distributed or flexible) BE
was studied in several prior/later works.
% It's an open problem to construct unbounded schemes
% from pairing with non-trivial storage efficiency.
Schemes based on
pairing~\cite{DHMR08,CCS:WQZD10,AC:KolMalWee23} or
% This construction is bounded because its security definition
% constrains "n" (binary, maximum number of recipients, given to Setup)
% to be polynomial.
% It's unclear to me whether the restriction can be removed with some effort,
% e.g., by considering "function-binding hash" for sparse strings.
witness encryption~\cite{C:FreWatWu23}
require global set-up, and
% This construction is bounded because of its syntax/security definition.
the obfuscation-based one~\cite{C:BonZha14} do not.
% However, nowhere else in this paper did I mention the distinction between
% bounded and unbounded schemes, so I am not mentioning it here either.

\paragraph{Broadcast, Trace, and Revoke.}
BTR~\cite{FC:NaoPin00,C:NaoNaoLot01} is also known as
\emph{broadcast and trace} or \emph{trace and revoke}.
Non-AH version of BTR supporting \emph{public tracing} with optimal size
currently is only known from
functional encryption for general circuits
(polynomial hardness, same as in this work)~\cite{EC:AKYY23,EC:JaiLinLuo23}
or witness encryption or obfuscation
(non-falsifiable assumptions)~\cite{EC:NisWicZha16,PKC:GoyVusWat19}.
BTR is also known
from pairing (standard assumption)~\cite{CCS:BonWat06,CCS:GKSW10}
with $\bigTheta(\sqrt{N})$-size components supporting \emph{public} tracing,
or from pairing (generic group model)~\cite{C:Zhandry20}
with various size trade-offs supporting \emph{secret} tracing
(still $\bigOmega(\sqrt{N})$ when size is balanced across components),
or from both pairing (standard assumption) and LWE~\cite{C:GQWW19}
with $\bigO(N^\epsilon)$-size ciphertexts for any constant ${\epsilon>0}$
but having $\bigOmega(N)$-size keys and only supporting \emph{secret} tracing.
Regardless of public or secret traceability,
these schemes generate recipients' decryption keys
correlated by the master secret key,
the major downside that AH-BTR intends to address.

Continuing the discussion of technical challenges in open questions,
AH-BTR implies BTR supporting \emph{public} tracing
with the same ciphertext size.
Therefore, it makes more sense to survey
the techniques for public-tracing BTR schemes,
than secret-tracing or non-BTR traitor tracing ones,
in search of non-obfustopia instantiations.
Filtered as such,
the only schemes~\cite{CCS:BonWat06,CCS:GKSW10}
with non-trivial (i.e., sublinear) ciphertext size
are pairing-based and
heavily rely on shared public parameter for key correlation
(enabling ciphertext compression)
---
antithetical to the fully decentralized nature of AH-BTR.
Even if we cease
the insistence of having no centrally generated public parameters,
the only known \ad hoc type of (pairing-based) schemes are
broadcast encryption~\cite{CCS:WQZD10,AC:KolMalWee23} without tracing,
which are clever modifications of non-AH BE schemes.
However, it is unclear
how such adaptations can be done for~\cite{CCS:BonWat06,CCS:GKSW10}
or, more generally,
how AH-BTR can be constructed following any known pairing-based paradigms.

\paragraph{Registration-Based (Registered) Encryption.}
RBE~\cite{TCC:GHMR18,PKC:GHMRS19,C:GoyVus20,IMA:ConEldSma21,CCS:GKMR23,EC:HLWW23,C:FreWatWu23,AC:FioKolPer23,AC:FFMMRV23,AC:ZZGQ23,EC:ZLZGQ24,C:GLWW24}
is an emerging paradigm to decentralize functional encryption,
where users generate their own key pairs
and
their public keys are aggregated for use during encryption.
AH-BTR and RBE share similarities in
motivation and techniques ---
e.g.,
typical constructions of both rely on laconic cryptography~\cite{C:CDGGMP17}
to compress public keys.

We can conceive casting \ad hoc private linear broadcast encryption,
our building block of AH-BTR,
as RBE for compare-index-and-reveal,
yet there is no study of this functionality in the literature.
Even under this view,
RBE is not ``ergonomic'' to the usage pattern of AH-BTR
and
such reduction may suffer efficiency issues.
The reason is that RBE requires
distributing decryption updates
(public information that, when used with user-generated secret keys,
helps with decrypting ciphertexts encrypted using the aggregated public key)
as the public keys are aggregated.
RBE aggregation corresponds to the choice of recipients in AH-BTR,
which happens at encryption time.
Consequently,
decryption updates from RBE would have to be included in every ciphertext,
or
every recipient must redo aggregation.
Without further investigation,
it is unclear whether the issue can be resolved.
In this work,
we study AH-BTR directly and do not try casting it under RBE.

\paragraph{Efficiency Parameters.}
Existing works studying
BE~\cite{C:FiaNao93,C:BonGenWat05,EC:GenWat09,C:BonWatZha14,EC:AgrYam20,TCC:AgrWicYam20,EPRINT:BraVai20,C:Zhandry20,EC:Wee22}
and
its extensions~\cite{PAIRING:DelPaiPoi07,AC:Delerablee07,EPRINT:SakFur07,C:BonZha14}
have been focused on improving the sizes of various components,
and the time complexity has been largely overlooked.
In a rare exception,
the work of~\cite{PKC:AttLib10}
reduces the number of pairing operations during decryption down to constant,
yet the overall decryption time is not among its concerns.
This work brings the total decryption time into the picture.

\paragraph{Lower Bounds.}
Previous works~\cite{EC:BluCre94,EC:LubSta98,AC:KYDB98,AFRICACRYPT:AusKre08,AC:KatYer09,C:GayKerWee15,ITC:DLY21} show a few efficiency lower bounds related to ABE and BE,
yet they only apply to information-theoretically secure primitives and even specific construction techniques.
Moreover, all of them prove space (ciphertext or secret key size, or their trade-off) lower bounds, whereas
this work is about space-time trade-offs.
Based on obfuscation~\cite{C:BonWatZha14} or both LWE and pairing~\cite{EC:AgrYam20}, broadcast encryption with ${|\ct|,|\sk|=\bigO(1)}$ can be achieved,
circumventing all previously known bounds.
A concurrent work~\cite{EC:JaiLinLuo23}
proves lower bounds
on (partially hiding) functional encryption,
which is more expressive than BE and ABE and
hence subject to stricter lower bounds than that in this work.
The two works complement each other
in understanding the efficiency trade-offs of advanced forms of encryption.
