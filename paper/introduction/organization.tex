\subsubsection{Organization.}
In \Section~\ref{sec:overview}, we provide an overview of our results.
In \Section~\ref{sec:preliminaries}, we present the preliminaries.%
\footnote{The technical parts of the preliminaries (those beyond the opening paragraphs) are only needed for the constructions.
% "Sections" instead of "\Sections" is correct (for Springer)
% because Springer uses abbreviations
% if and only if the word does not lead a sentence.
% The same applies to the sentence that is commented out.
Sections~\ref{sec:ahbtr-definitions} and~\ref{sec:lower-bound} do not depend on them.}
In \Section~\ref{sec:ahbtr-definitions}, we formally define \ad hoc broadcast, trace, and revoke (AH-BTR) and its security notions, and prove the relation among them.
\luoji{The proof is likely not in LNCS format.}
In \Section~\ref{sec:ahplbe}, we define \ad hoc private linear broadcast encryption (AH-PLBE), an intermediate object for constructing AH-BTR, and construct such a scheme.
In \Section~\ref{sec:ahbtr-from-ahplbe}, we present the construction of AH-BTR from AH-PLBE.
In \Section~\ref{sec:ahbtr-trade-off}, we show how to trade ciphertext size for decryption time in AH-BTR.
In \Section~\ref{sec:lower-bound}, we prove the lower bound of the trade-offs between ciphertext size and decryption time.
% Section~\ref{sec:symbols} contains a table of non-descriptive symbols for reference.
