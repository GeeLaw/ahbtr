\subsubsection{Open Questions.}
The tracing model in this work is black-box and classical, and recent works~\cite{C:Zhandry21,TCC:Zhandry20} have studied white-box or quantum traitor tracing.
Conceptually, it is interesting to understand the \ad hoc versions of those tracing models.

Another question for future investigation is whether
(weakened versions of) AH-BTR can be constructed from more lightweight assumptions such as factoring-related, group-based, or lattice-based assumptions.
% Interestingly, our lower bound still holds even if
% it is a bounded scheme statically secure against $1$-bounded collusion,
% again as a corollary of the ABE bound.
Potential weakening include
\begin{itemize}
\item making the scheme bounded,%
\footnote{A maximum of number of recipients per ciphertext is set when generating a key pair, and only ``compatible'' public keys can be used to form a recipient set.}
% Due to the simplified security notions,
% there's no version between "very selective" (static) and "adaptive".
\item settling for static security,
\item considering security against bounded collusion, and
\item only achieving threshold tracing~\cite{C:NaoPin98}.
\end{itemize}
