\subsubsection{\scalebox{0.8}[1]{{\fontfamily{ptm}\bfseries\selectfont YOU CAN (NOT) OPTIMIZE\@.}}}
% This subsection title is a tribute to "Evangerion Shin Gekijouban".
While our first construction enjoys constant-size ciphertext,
its decryption algorithm runs in time~$\bigOmega(N)$.
Concretely, the laconic OT hash is a Merkle tree, and before performing laconic OT decryption, it is necessary to reconstruct the tree as it is not stored in the ciphertext.
In contrast, the decryption time of the scheme implied by the na{\"i}ve PLBE is constant in the RAM model, as it only looks at the relevant piece of the underlying PKE ciphertext.

We can trade ciphertext size for decryption time by using the na{\"i}ve PLBE on top of our construction.
By grouping the recipients into $\bigTheta(N^{1-\gamma})$ sets of size~$\bigTheta(N^\gamma)$ and using our basic construction over each set, we obtain a scheme with ciphertext size~$\bigTheta(N^{1-\gamma})$ and decryption time~$\bigTheta(N^\gamma)$.
The core idea of this transformation was formalized as the user expansion compiler~\cite{C:Zhandry20} in the context of traditional traitor tracing.

All the constructions we now know have ${|\ct|\cdot T_\Dec=\bigOmega(N)}$,
where $|\ct|$ is the ciphertext length and $T_\Dec$ is the decryption time.
It turns out that this bound necessarily holds for all secure AH-BTR, and
the blame is on the functionality of broadcast encryption (not traitor tracing).
% I haven't verified whether TT from LWE has short decryption time,
% so no citation at this moment.
Indeed, 
it is possible to make both $|\ct|$ and $T_\Dec$ constant in a traditional traitor tracing scheme~\cite{C:BonZha14}.
In existing broadcast encryption (or revocation) schemes~\cite{C:BonGenWat05,AC:Delerablee07,EC:GenWat09,C:BonZha14,EC:AgrYam20,TCC:AgrWicYam20,EPRINT:BraVai20} for $N$ users,
encrypting to arbitrary subsets of size~$S$ or $({N-S})$ makes ${|\ct|\cdot T_\Dec=\bigOmega(S)}$.
It is precisely the capability to encrypt to many ${(N/2)}$-subsets among $N$ users that is the deal breaker, as we shall see in the formal proof.
Interestingly,
the adversary used in the proof
simply runs the decryption algorithm with a \emph{non-decrypting} key
(while \emph{lying} about the recipient set),
so the bound holds as long as the scheme is not \emph{blatantly} insecure.

We explain the ideas of our proof based on a corollary%
\footnote{This corollary is also a lower bound of a probabilistic variant of Yao's box problem~\cite{STOC:Yao90} (generalized and studied in~\cite{EC:CorHenKog22}), on which our proof can be alternatively based.
% In the proof based on Yao's box problem, to recover R[i],
% try decryption twice with R[i] being 0 or 1.
% Correctness and security guarantee that decryption recovers
% the correct message only when R[i] is set to the right value.
% To directly use \cite{STOC:Yao90} result, we do the following:
% 1. Use correctness-security gap to obtain a BPP-type algorithm
%    determining R[i] (let's say that the algorithm only has a
%    small inverse-polynomial [in lambda] gap of success).
% 2. Use BPP amplification to obtain BPP-type algorithm with
%    exponentially small failure probability, after a fixed
%    polynomial (in lambda) number of attempts.
% 3. Use counting argument to show that there exists a randomness
%    that perfectly recovers R[i].
% 4. Use \cite{STOC:Yao90} bound to conclude the desired inequality.
% This bound will look like
%     |ct| * T_Dec >= N / poly(lambda)
% for some fixed polynomial.
}
of a result~\cite{C:Unruh07} dealing with random oracles in the presence of non-uniform advice.
Let ${S,T\geq 0}$ be such that ${ST\ll N}$.
The corollary says that
for any adversary learning any $S$-bit function (advice) of a random string~${R\draws\bit^N}$ and additionally (adaptively) querying at most $T$ bits in~$R$,
it is ``indistinguishable'' to flip a bit in~$R$ at a random location
after the advice is computed (using the non-flipped~$R$) and
before queries are answered,
even if the index of the potentially flipped bit is revealed to the adversary after the advice is computed.

Back to AH-BTR\@. Imagine that there are $2N$ users in the system, associated with key pairs $(\pk_{j,s},\sk_{j,s})$ for ${j\in[N]}$ and ${s\in\bit}$.
Consider a ciphertext~$\ct$ encrypting a random plaintext to $\{\pk_{j,R[j]}\}_{j\in[N]}$ for a random string~$R$ and
regard $\ct$ as the advice.
Suppose $Y$ is either $R$ itself or $R$ flipped at index ${i^\ast\draws[N]}$.
Let's try decrypting~$\ct$ using $\sk_{i^\ast,Y[i^\ast]}$
while \emph{pretending} that $\ct$ is generated for~$Y$.
Each time the AH-BTR decryption algorithm wants to read~$\pk_j$,
we probe $Y[j]$ and respond with~$\pk_{j,Y[j]}$.
By way of contradiction, suppose ${|\ct|\cdot T_\Dec\ll N}$,
which would translate to ${ST\ll N}$ in the corollary.

By the correctness of AH-BTR, when $Y$ is $R$ itself,
the attempted decryption should successfully recover the plaintext.
From the corollary it follows that the other case
($Y$ is $R$ flipped at~$i^\ast$)
should also lead to successful recovery.
But if ${Y[i^\ast]=\neg R[i^\ast]}$,
by the security of AH-BTR,
the attempted decryption must fail to recover the plaintext except for negligible probability, yielding a contradiction.
