\subsubsection{You Can (Not) Optimize.}
While our construction achieves constant-size ciphertext,
its decryption algorithm runs in time~$\Omega(N)$.
In contrast, the decryption time of the na{\"i}ve PLBE is constant in the RAM model, as it only looks at the relevant piece of the underlying PKE ciphertext.

We can trade ciphertext size for decryption time by using the na{\"i}ve PLBE on top of our construction.
By grouping the recipients into $N^{1-\gamma}$ sets of size~$N^\gamma$ and using our basic construction inside each set, we obtain a scheme with ciphertext size~$\Theta(N^{1-\gamma})$ and decryption time~$\Theta(N^\gamma)$.
This transformation was formalized as the user expansion compiler~\cite{C:Zhandry20} in the context of traditional traitor tracing.

All the constructions we now know have ${|\ct|\cdot\TDec=\Omega(N)}$,
where $|\ct|$ is the ciphertext length and $\TDec$ is the decryption time.
Concretely, for our basic construction, the LOT hash is a Merkle tree, and before performing LOT decryption, it is necessary to reconstruct the tree as it is not stored in the ciphertext.
If it is stored, the decryption time goes down to constant at the cost of bumping up the ciphertext size to~$\Omega(N)$.

It turns out that such bound necessarily holds for all secure AH-BTR, and
it is due to the functionality of broadcast encryption (not traitor tracing).
\luoji{Pending citations.}
Indeed, existing broadcast encryption schemes all have such limitation, whereas
\luoji{Pending citation.}
it is possible to make both $|\ct|$ and $\TDec$ constant in a traditional traitor tracing scheme.

We explain the intuition%
\footnote{The intuition is highly informal and contains \href{https://twitter.com/booleananalysis/status/1219821061430300676}{street-fighting maths}.}
for the special case of constant-size ciphertexts, for which we want to show ${\TDec=\Omega(N)}$.
To begin, imagine there are $2N$ users in the system, associated with key pairs $(\pk_{j,b},\sk_{j,b})$ for ${j\in[N]}$ and ${b\in\bit}$.
Consider a ciphertext encrypting a random plaintext to $\{\pk_{j,R[j]}\}_{j\in[N]}$ for a random string~$R$ and
regard this ciphertext as a (randomized) hash of~$R$, as illustrated in \Figure~\ref{fig:lower-bound}.

\begin{figure}
\centering
\capstart
\begingroup
\definecolor{MyGrey}{RGB}{160,160,160}
\definecolor{MyRed}{RGB}{220,16,16}
\def\MakeGrey#1{{\color{MyGrey}#1}}%
\def\MakeRed#1{{\color{MyRed}#1}}%
\fbox{\hspace*{.8em}$\displaystyle
\begin{aligned}
\textup{encrypt }\ct
\begin{pmatrix}
\pk_{1,0}&\MakeGrey{\pk_{2,0}}&\pk_{3,0}&\MakeGrey{\pk_{4,0}}&\MakeGrey{\pk_{5,0}}\\
\MakeGrey{\pk_{1,1}}&\pk_{2,1}&\MakeGrey{\pk_{3,1}}&\pk_{4,1}&\pk_{5,1}
\end{pmatrix}
&{}\Longleftrightarrow
\textup{hash of }R=01011
\rule[2em]{0pt}{0pt}
\rule[-1.4em]{0pt}{0pt}
\\
\textup{decrypt }\ct
\begin{pmatrix}
\pk_{1,0}&\MakeGrey{\pk_{2,1}}&
\begin{aligned}\MakeRed{\pk_{3,1}}&\\\MakeRed{\sk_{3,1}}&\end{aligned}&
\pk_{4,1}&\MakeGrey{\pk_{5,1}}
\end{pmatrix}
&{}\Longleftrightarrow
\hphantom{\textup{hash of }}
\mathllap{\textup{verify }}
R\overset{?}{=}0{\MakeGrey\ast}\MakeRed11{\MakeGrey\ast}
\left\{\begin{aligned}
0\MakeGrey0\MakeRed11\MakeGrey0&\ 0\MakeGrey0\MakeRed11\MakeGrey1\\
0\MakeGrey1\MakeRed11\MakeGrey0&\ 0\MakeGrey1\MakeRed11\MakeGrey1
\end{aligned}\right.
\rule[-1.6em]{0pt}{0pt}
\end{aligned}
$\hspace*{.8em}}%
\par\begin{minipage}{32em}
\caption{An example in the intuition of the lower bound with
${N=5,i^\ast=3}$.\\
\WideNarrow{\hspace*{4em}}{\hspace*{2em}}
(Legends: encrypt --- \MakeGrey{not selected}; decrypt --- \MakeGrey{not probed}, \MakeRed{flipped}.)}
\end{minipage}
\endgroup

\label{fig:lower-bound}
\end{figure}

Intuitively, the hash should be collision-resistant, i.e., it should computationally bind~$R$, as only $\sk_{j,R[j]}$'s and not $\sk_{j,\neg R[j]}$'s are supposed to decrypt the ciphertext.
Furthermore, with an attempted decryption using~$\sk_{i,b}$,
\luoji{This reminds me of the witness encryption / obfuscation perspective of laconic OT.}
the decryptor implicitly verifies whether the hash is bound to some $R$ with ${R[i]=b}$, because decryption should succeed or fail depending on ${R[i]\overset{?}{=}b}$.
Since the hash is short, it cannot perfectly bind~$R$, and essentially, the only way to verify this binding relation is to recompute the hash from~$R$ and compare, as argued below.

By way of contradiction, suppose ${\TDec=\smallo(N)}$, then
the decryption algorithm would probe at most $\smallo(N)$ keys among ${\{\pk_j=\pk_{j,R[j]}\}_{j\in[N]}}$.
We say ${R'\in\bit^N}$ is a ``candidate'' if it agrees with $R$ on all indices~$j$ for which $\pk_j$ is probed during decryption.

Given a ciphertext~$\ct$, we choose a random location~$i^\ast$,
pretend that the $(i^\ast)$\textsuperscript{th} bit of~$R$ were ${\neg R[i^\ast]}$ (i.e., pretend $\pk_{i^\ast}$ were $\pk_{i^\ast,\neg R[i^\ast]}$), and try decrypting using $\sk_{i^\ast,\neg R[i^\ast]}$.
Due to the security of the scheme, such attempt should (almost always) fail to recover the plaintext.

On the other hand, by the correctness of the scheme, if $\ct$ is a possible ciphertext (i.e., produced by the encryption algorithm with positive probability) for any candidate~$R'$, the decryption must succeed.
The ciphertext is of constant length, and there would be ${2^{(1-\smallo(1))N}\ggg 2^{|\ct|}}$ candidates, so it is quite likely that $\ct$ would indeed be possible for some candidate, leading to a contradiction.

The formal and general proof uses a result~\cite{C:Unruh07} dealing with random oracles in the presence of non-uniform advice.
