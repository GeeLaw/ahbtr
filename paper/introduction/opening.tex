Traitor tracing schemes~\cite{C:ChoFiaNao94} enable content distributors to fight against piracy.
A content distributor such as a media streaming service can generate a public key and many different secret keys for individual subscribers,
all of which can decrypt the ciphertexts created using the public key.
Given a pirate decoder capable of decrypting,
which could have been created from the secret keys of multiple subscribers,
the tracing algorithm can find at least one subscriber (a traitor) whose key was used to create the said decoder.

While the traditional model caters well to the needs of content distributors,
its centralized nature makes it unsuitable for many scenarios,
e.g., when a group of individuals want to communicate amongst themselves and trace traitors who provide decoders to outsiders.%
\footnote{For a concrete scenario, see~\cite{C:Zhandry21}.}

Can we instead allow each party to generate its own public and secret key pair?
The encryption algorithm is given a set of public keys, and the ciphertext can be decrypted by any of the corresponding secret keys.
The tracing algorithm is given a set~$S$ of public keys and a decoder that is capable of decrypting ciphertexts encrypted to~$S$, and it must identify at least one public key whose secret key has been used to construct the decoder.
It turns out that there are many subtle issues in defining such a scheme to accurately capture the desired security goal.
