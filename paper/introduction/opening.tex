Traitor tracing schemes~\cite{C:ChoFiaNao94} enable content distributors to fight against piracy.
A content distributor such as a media streaming service can generate a public key and many different secret keys for individual subscribers,
all of which can decrypt the ciphertexts created using the public key.
Given a pirate decoder capable of decrypting,
which could have been created from the secret keys of multiple subscribers,
the tracing algorithm can find at least one subscriber (a traitor) whose key was used to create the said decoder.

While the traditional model caters well to the needs of content distributors,
its centralized nature makes it unsuitable for many scenarios,
e.g., when a group of individuals want to communicate amongst themselves and trace traitors who provide decoders to outsiders. (See~\cite{C:Zhandry21} for a more concrete example.)

The first question is thus naturally the following:
\begin{center}
\itshape
What is the right notion of secure ad hoc traitor tracing schemes?\\
\end{center}
Having formalized the syntax and security of such schemes,
we study its constructions:
\begin{center}
\itshape
How can such a scheme be constructed,\\
from what assumptions and with what efficiency?
\end{center}
Efficiency improvement never ends until we reach the optimum,
for which it is necessary to understand where the limit stands:
\begin{center}
\itshape
What bounds are there on the efficiency of such schemes?
\end{center}
