\subsubsection{Our Contributions.}
We provide answers to all three questions:
\begin{itemize}
\item \emph{Conceptually}, we pose the question of \ad hoc traitor tracing,
develop from the ideas thereof, and
eventually reach the definitions for \ad hoc broadcast, trace, and revoke (AH-BTR).
We prove the relation among the security notions considered in this work.
\item \emph{Construction-wise},
we present a secure AH-BTR based on functional encryption for general circuits.
Ignoring polynomial factors in the security parameter,
our scheme achieves
\begin{align*}
\textup{encryption time}\qquad\qquad\mathllap{T_\Enc}&{}=\bigO(N),\\
\textup{ciphertext size}\qquad\qquad\mathllap{|\ct|}&{}=\bigO(N^{1-\gamma}),\\
\textup{decryption time}\qquad\qquad\mathllap{T_\Dec}&{}=\bigO(N^\gamma),
\end{align*}
for any constant ${0\leq\gamma\leq 1}$.
\item \emph{Questing for the ultimate efficiency},
we prove that for all secure AH-BTR,
\begin{align*}
|\ct|\cdot\TDec=\bigOmega(N),
\end{align*}
so our scheme offers all possible optimal trade-offs between $|\ct|$ and~$\TDec$.
Better yet, the bound holds for a specific (weak) attribute-based encryption (ABE) scheme, thus also shedding new insights into the efficiency of ABE.
\end{itemize}
A final addition is that our scheme is \emph{compatible} with the existing public-key infrastructure,
i.e., the keys of such a scheme can be those of any secure public-key encryption, and
there is no need to regenerate keys to take advantage of our scheme.
