\subsubsection{Our Contributions.}
We provide answers to all three questions:
\begin{itemize}
\item \emph{Conceptually}, we pose the question of \ad hoc traitor tracing,
develop from the ideas thereof, and
eventually arrive at the definitions for \ad hoc broadcast, trace, and revoke (AH-BTR).
We prove the relation among the security notions considered in this work.
\item \emph{Construction-wise},
we present secure AH-BTR schemes based on functional encryption for general circuits~\cite{TCC:BonSahWat11}.
With polynomial factors in the security parameter ignored,
they achieve
\begin{align*}
\textup{encryption time}\qquad\qquad\mathllap{T_\Enc}&{}=\bigO(N),\\
\textup{ciphertext size}\qquad\qquad\mathllap{|\ct|}&{}=\bigO(N^{1-\gamma}),\\
\textup{decryption time}\qquad\qquad\mathllap{T_\Dec}&{}=\bigO(N^\gamma),
\end{align*}
for any constant ${0\leq\gamma\leq 1}$,
where $N$ is the number of recipients.
\item \emph{Questing for the ultimate efficiency},
we prove that for all secure AH-BTR,
\begin{align*}
|\ct|\cdot T_\Dec=\bigOmega(N),
\end{align*}
so our schemes offer all possible optimal trade-offs between $|\ct|$ and~$T_\Dec$,
fully demonstrating the Pareto front of AH-BTR efficiency.
Better yet, the bound holds for a restricted kind of weakly secure broadcast encryption~\cite{C:FiaNao93},
which is a specific case of attribute-based encryption~\cite{EC:SahWat05,CCS:GPSW06}.
Our result is the first space-time lower bound applicable to any computationally secure BE scheme,
shedding new insights into the efficiency of ABE and BE.
\end{itemize}
A final addition is that our scheme is \emph{compatible} with the existing public-key encryption schemes,
i.e., the keys of such a scheme can be those of any secure public-key encryption, and
there is no need to regenerate keys to take advantage of our scheme.
